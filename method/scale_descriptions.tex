% Options for packages loaded elsewhere
\PassOptionsToPackage{unicode}{hyperref}
\PassOptionsToPackage{hyphens}{url}
\PassOptionsToPackage{dvipsnames,svgnames,x11names}{xcolor}
%
\documentclass[
  letterpaper,
  DIV=11,
  numbers=noendperiod]{scrartcl}

\usepackage{amsmath,amssymb}
\usepackage{lmodern}
\usepackage{iftex}
\ifPDFTeX
  \usepackage[T1]{fontenc}
  \usepackage[utf8]{inputenc}
  \usepackage{textcomp} % provide euro and other symbols
\else % if luatex or xetex
  \usepackage{unicode-math}
  \defaultfontfeatures{Scale=MatchLowercase}
  \defaultfontfeatures[\rmfamily]{Ligatures=TeX,Scale=1}
\fi
% Use upquote if available, for straight quotes in verbatim environments
\IfFileExists{upquote.sty}{\usepackage{upquote}}{}
\IfFileExists{microtype.sty}{% use microtype if available
  \usepackage[]{microtype}
  \UseMicrotypeSet[protrusion]{basicmath} % disable protrusion for tt fonts
}{}
\makeatletter
\@ifundefined{KOMAClassName}{% if non-KOMA class
  \IfFileExists{parskip.sty}{%
    \usepackage{parskip}
  }{% else
    \setlength{\parindent}{0pt}
    \setlength{\parskip}{6pt plus 2pt minus 1pt}}
}{% if KOMA class
  \KOMAoptions{parskip=half}}
\makeatother
\usepackage{xcolor}
\setlength{\emergencystretch}{3em} % prevent overfull lines
\setcounter{secnumdepth}{-\maxdimen} % remove section numbering
% Make \paragraph and \subparagraph free-standing
\ifx\paragraph\undefined\else
  \let\oldparagraph\paragraph
  \renewcommand{\paragraph}[1]{\oldparagraph{#1}\mbox{}}
\fi
\ifx\subparagraph\undefined\else
  \let\oldsubparagraph\subparagraph
  \renewcommand{\subparagraph}[1]{\oldsubparagraph{#1}\mbox{}}
\fi


\providecommand{\tightlist}{%
  \setlength{\itemsep}{0pt}\setlength{\parskip}{0pt}}\usepackage{longtable,booktabs,array}
\usepackage{calc} % for calculating minipage widths
% Correct order of tables after \paragraph or \subparagraph
\usepackage{etoolbox}
\makeatletter
\patchcmd\longtable{\par}{\if@noskipsec\mbox{}\fi\par}{}{}
\makeatother
% Allow footnotes in longtable head/foot
\IfFileExists{footnotehyper.sty}{\usepackage{footnotehyper}}{\usepackage{footnote}}
\makesavenoteenv{longtable}
\usepackage{graphicx}
\makeatletter
\def\maxwidth{\ifdim\Gin@nat@width>\linewidth\linewidth\else\Gin@nat@width\fi}
\def\maxheight{\ifdim\Gin@nat@height>\textheight\textheight\else\Gin@nat@height\fi}
\makeatother
% Scale images if necessary, so that they will not overflow the page
% margins by default, and it is still possible to overwrite the defaults
% using explicit options in \includegraphics[width, height, ...]{}
\setkeys{Gin}{width=\maxwidth,height=\maxheight,keepaspectratio}
% Set default figure placement to htbp
\makeatletter
\def\fps@figure{htbp}
\makeatother
\newlength{\cslhangindent}
\setlength{\cslhangindent}{1.5em}
\newlength{\csllabelwidth}
\setlength{\csllabelwidth}{3em}
\newlength{\cslentryspacingunit} % times entry-spacing
\setlength{\cslentryspacingunit}{\parskip}
\newenvironment{CSLReferences}[2] % #1 hanging-ident, #2 entry spacing
 {% don't indent paragraphs
  \setlength{\parindent}{0pt}
  % turn on hanging indent if param 1 is 1
  \ifodd #1
  \let\oldpar\par
  \def\par{\hangindent=\cslhangindent\oldpar}
  \fi
  % set entry spacing
  \setlength{\parskip}{#2\cslentryspacingunit}
 }%
 {}
\usepackage{calc}
\newcommand{\CSLBlock}[1]{#1\hfill\break}
\newcommand{\CSLLeftMargin}[1]{\parbox[t]{\csllabelwidth}{#1}}
\newcommand{\CSLRightInline}[1]{\parbox[t]{\linewidth - \csllabelwidth}{#1}\break}
\newcommand{\CSLIndent}[1]{\hspace{\cslhangindent}#1}

\KOMAoption{captions}{tableheading}
\makeatletter
\makeatother
\makeatletter
\makeatother
\makeatletter
\@ifpackageloaded{caption}{}{\usepackage{caption}}
\AtBeginDocument{%
\ifdefined\contentsname
  \renewcommand*\contentsname{Table of contents}
\else
  \newcommand\contentsname{Table of contents}
\fi
\ifdefined\listfigurename
  \renewcommand*\listfigurename{List of Figures}
\else
  \newcommand\listfigurename{List of Figures}
\fi
\ifdefined\listtablename
  \renewcommand*\listtablename{List of Tables}
\else
  \newcommand\listtablename{List of Tables}
\fi
\ifdefined\figurename
  \renewcommand*\figurename{Figure}
\else
  \newcommand\figurename{Figure}
\fi
\ifdefined\tablename
  \renewcommand*\tablename{Table}
\else
  \newcommand\tablename{Table}
\fi
}
\@ifpackageloaded{float}{}{\usepackage{float}}
\floatstyle{ruled}
\@ifundefined{c@chapter}{\newfloat{codelisting}{h}{lop}}{\newfloat{codelisting}{h}{lop}[chapter]}
\floatname{codelisting}{Listing}
\newcommand*\listoflistings{\listof{codelisting}{List of Listings}}
\makeatother
\makeatletter
\@ifpackageloaded{caption}{}{\usepackage{caption}}
\@ifpackageloaded{subcaption}{}{\usepackage{subcaption}}
\makeatother
\makeatletter
\@ifpackageloaded{tcolorbox}{}{\usepackage[many]{tcolorbox}}
\makeatother
\makeatletter
\@ifundefined{shadecolor}{\definecolor{shadecolor}{rgb}{.97, .97, .97}}
\makeatother
\makeatletter
\makeatother
\ifLuaTeX
  \usepackage{selnolig}  % disable illegal ligatures
\fi
\IfFileExists{bookmark.sty}{\usepackage{bookmark}}{\usepackage{hyperref}}
\IfFileExists{xurl.sty}{\usepackage{xurl}}{} % add URL line breaks if available
\urlstyle{same} % disable monospaced font for URLs
\hypersetup{
  pdftitle={Scale Descriptions},
  colorlinks=true,
  linkcolor={blue},
  filecolor={Maroon},
  citecolor={Blue},
  urlcolor={Blue},
  pdfcreator={LaTeX via pandoc}}

\title{Scale Descriptions}
\author{}
\date{}

\begin{document}
\maketitle
\ifdefined\Shaded\renewenvironment{Shaded}{\begin{tcolorbox}[borderline west={3pt}{0pt}{shadecolor}, enhanced, sharp corners, interior hidden, breakable, boxrule=0pt, frame hidden]}{\end{tcolorbox}}\fi

\hypertarget{mini-ipip-6}{%
\paragraph{Mini-IPIP 6}\label{mini-ipip-6}}

We measured participants personality with the Mini International
Personality Item Pool 6 (Mini-IPIP6) (C. G. Sibley et al., 2011) which
consists of six dimensions and each dimensions is measured with four
items:

\begin{enumerate}
\def\labelenumi{\arabic{enumi}.}
\item
  agreeableness,

  \begin{enumerate}
  \def\labelenumii{\roman{enumii}.}
  \item
    I sympathize with others' feelings.
  \item
    I am not interested in other people's problems. (r)
  \item
    I feel others' emotions.
  \item
    I am not really interested in others. (r)
  \end{enumerate}
\item
  conscientiousness,

  \begin{enumerate}
  \def\labelenumii{\roman{enumii}.}
  \item
    I get chores done right away.
  \item
    I like order.
  \item
    I make a mess of things. (r)
  \item
    I ften forget to put things back in their proper place. (r)
  \end{enumerate}
\item
  extraversion,

  \begin{enumerate}
  \def\labelenumii{\roman{enumii}.}
  \item
    I am the life of the party.
  \item
    I don't talk a lot. (r)
  \item
    I keep in the background. (r)
  \item
    I talk to a lot of different people at parties.
  \end{enumerate}
\item
  honesty-humility,

  \begin{enumerate}
  \def\labelenumii{\roman{enumii}.}
  \item
    I feel entitled to more of everything. (r)
  \item
    I deserve more things in life. (r)
  \item
    I would like to be seen driving around in a very expensive car. (r)
  \item
    I would get a lot of pleasure from owning expensive luxury goods.
    (r)
  \end{enumerate}
\item
  neuroticism, and

  \begin{enumerate}
  \def\labelenumii{\roman{enumii}.}
  \item
    I have frequent mood swings.
  \item
    I am relaxed most of the time. (r)
  \item
    I get upset easily.
  \item
    I seldom feel blue. (r)
  \end{enumerate}
\item
  openness to experience

  \begin{enumerate}
  \def\labelenumii{\roman{enumii}.}
  \item
    I have a vivid imagination.
  \item
    I have difficulty understanding abstract ideas. (r)
  \item
    I do not have a good imagination. (r)
  \item
    I am not interested in abstract ideas. (r)
  \end{enumerate}
\end{enumerate}

Each dimension was assessed with four items and participants rated the
accuracy of each item as it applies to them from 1 (Very Inaccurate) to
7 (Very Accurate). Items marked with (r) are reverse coded.

\hypertarget{age}{%
\paragraph{Age}\label{age}}

We asked participants' age in an open-ended question (``What is your
age?'').

\hypertarget{alcohol-frequency}{%
\paragraph{Alcohol Frequency}\label{alcohol-frequency}}

We measured participants' frequency of drinking alcohol using one item
adapted from Health (2013). Participants were asked ``How often do you
have a drink containing alcohol?'' (1 = Never - I don't drink, 2 =
Monthly or less, 3 = Up to 4 times a month, 4 = Up to 3 times a week, 5
= 4 or more times a week, 6 = Don't know).

\hypertarget{alcohol-intensity}{%
\paragraph{Alcohol Intensity}\label{alcohol-intensity}}

We measured participants' intensity of drinking alcohol using one item
adapted from Health (2013). Participants were asked ``How many drinks
containing alcohol do you have on a typical day when drinking alcohol?
(number of drinks on a typical day when drinking)''

\hypertarget{body-satisfaction}{%
\paragraph{Body Satisfaction}\label{body-satisfaction}}

We measured body satisfaction with one item from Stronge et al. (2015):
``I am satisfied with the appearance, size and shape of my body'', which
participants rated from 1 (very inaccurate) to 7 (very accurate).

\hypertarget{nz-born}{%
\paragraph{NZ-Born}\label{nz-born}}

We asked participants ``Which country were you born in?''.

\hypertarget{belief-in-god}{%
\paragraph{Belief in God}\label{belief-in-god}}

Using one item from Eurobarometer (2005), we asked participants ``Do you
believe in a God'' (1 = Yes, 0 = No).

\hypertarget{belief-in-sprituality}{%
\paragraph{Belief in Sprituality}\label{belief-in-sprituality}}

Using one item from Eurobarometer (2005), we asked participants ``Do you
believe in some form of spirit or lifeforce? (1 = Yes, 0 = No).

\hypertarget{felt-belongingness}{%
\paragraph{Felt Belongingness}\label{felt-belongingness}}

We assessed felt belongingness with three items adapted from the Sense
of Belonging Instrument (Hagerty \& Patusky, 1995): (1) ``Know that
people in my life accept and value me''; (2) ``Feel like an outsider'';
(3) ``Know that people around me share my attitudes and beliefs''.
Participants responded on a scale from 1 (Very Inaccurate) to 7 (Very
Accurate). The second item was reversely coded.

\hypertarget{charity-donation}{%
\paragraph{Charity Donation}\label{charity-donation}}

Using one item from Hoverd \& Sibley (2010), we asked participants ``How
much money have you donated to charity in the last year?''. To stablise
this indicator, we first took the natural log of the response + 1, and
then centred and standardised the log-transformed indicator.

\hypertarget{number-of-children}{%
\paragraph{Number of Children}\label{number-of-children}}

We measured number of children using one item from Bulbulia (2015). We
asked participants ``How many children have you given birth to,
fathered, or adopted. How many children have you given birth to,
fathered, or adopted?

\hypertarget{frequency-of-church-attendence}{%
\paragraph{Frequency of Church
Attendence}\label{frequency-of-church-attendence}}

If participants answered \emph{yes} to ``Do you identify with a religion
and/or spiritual group?'' we measured their frequency of church
attendence using one item from \&. B. Sibley C. G. (2012): ``how many
times did you attend a church or place of worship in the last month?''.

\hypertarget{sense-of-community}{%
\paragraph{Sense of Community}\label{sense-of-community}}

We measured sense of community with a single item from Sengupta et al.
(2013): ``I feel a sense of community with others in my local
neighbourhood.'' Participants answered on a scale of 1 (strongly
disagree) to 7 (strongly agree).

\hypertarget{education-attainment}{%
\paragraph{Education Attainment}\label{education-attainment}}

Participants were asked ``What is your highest level of
qualification?''. We coded participans highest finished degree according
to the New Zealand Qualifications Authority.

\hypertarget{employment}{%
\paragraph{Employment}\label{employment}}

We asked participants ``Are you currently employed? (This includes
self-employed or casual work)''.

\hypertarget{job-security}{%
\paragraph{Job Security}\label{job-security}}

Participants indicated their feeling of job security by answering ``How
secure do you feel in your current job?'' on a scale from 1 (not secure)
to 7 (very secure).

\hypertarget{emotional-regulation}{%
\paragraph{Emotional Regulation}\label{emotional-regulation}}

We measured participants' levels of emotional regulation using three
items adpated from Gratz \& Roemer (2004) and Gross \& John (2003): (1)
``When I feel negative emotions, my emotions feel out of control.''; (2)
``When I feel negative emotions, I suppress or hide my emotions.''; (3)
``When I feel negative emotions, I change the way I think to help me
stay calm.''. Participants were asked to indicate the extent to which
they agree with these items (1 = Strongly Disagree to 7 = Strongly
Agree).

\hypertarget{european}{%
\paragraph{European}\label{european}}

Participants were asekd ``Which ethnic group do you belong to (NZ census
question)?''. Europeans were coded as 1, wherease other ethniciities
were coded as 0.

\hypertarget{ethnicity}{%
\paragraph{Ethnicity}\label{ethnicity}}

Based on the New Zealand Cencus, we asked participants ``Which ethnic
group(s) do you belong to?''. The responses were: (1) New Zealand
European; (2) Māori; (3) Samoan; (4) Cook Island Māori; (5) Tongan; (6)
Niuean; (7) Chinese; (8) Indian; (9) Other such as DUTCH, JAPANESE,
TOKELAUAN. Please state:. We coded their answers into four groups:
Maori, Pacific, Asian, and Euro (except for Time 3, which used an
open-ended measure).

\hypertarget{gender}{%
\paragraph{Gender}\label{gender}}

We asked participants' gender in an open-ended question (``what is your
gender?''. Female was coded as 0, Male was coded as 1, and gender
diverse coded as 3 (Fraser et al., 2020).

\hypertarget{gratitude}{%
\paragraph{Gratitude}\label{gratitude}}

We assessed the extent to which participants have gratitude, using three
items from McCullough et al. (2002): (1) ``I have much in my life to be
thankful for.''; (2) ``When I look at the world, I don't see much to be
grateful for.''; (3) ``I am grateful to a wide variety of people.''.
Participants indicated their agreement with these items (1 = Strongly
Disagree to 7 = Strongly Agree). The second item was reversely coded.

\hypertarget{home-owner}{%
\paragraph{Home Owner}\label{home-owner}}

Using one item from Houkamau \& Sibley (2015), we asked participants
``(Māori) Do you own your own home?(either partly or fully owned)'' (1 =
Yes, 0 = No)

\hypertarget{hours-of-exercise}{%
\paragraph{Hours of Exercise}\label{hours-of-exercise}}

We measured hours of exercising using one item from C. G. Sibley et al.
(2011). We asked participants to estimate and report how many hours they
spend in exercise/physical activity last week. To stablise this
indicator, we first took the natural log of the response + 1, and then
centred and standardised the log-transformed indicator.

\hypertarget{hours-of-work}{%
\paragraph{Hours of Work}\label{hours-of-work}}

We measured hours of Work using one item from C. G. Sibley et al.
(2011). We asked participants to estimate and report how many hours they
spend in working in paid employment last week.

\hypertarget{body-mass-index}{%
\paragraph{Body Mass Index}\label{body-mass-index}}

Participants were asked ``What is your height? (metres)'' and ``What is
your weight? (kg)''. Based on participants indication of their height
and weight we calculated the BMI by dividing the weight in kilograms by
the square of the height in meters.

\hypertarget{disability}{%
\paragraph{Disability}\label{disability}}

We assessed disability with a one item indicator adapted from Verbrugge
(1997), that asks ``Do you have a health condition or disability that
limits you, and that has lasted for 6+ months?'' (1 = Yes, 0 = No).

\hypertarget{fatigue}{%
\paragraph{Fatigue}\label{fatigue}}

We assessed subjective fatigue by asking participants, ``During the last
30 days, how often did \ldots{} you feel exhausted?'' Responses were
collected on an ordinal scale (0 = None of The Time, 1 = A little of The
Time, 2 = Some of The Time, 3 = Most of The Time, 4 = All of The Time).

\hypertarget{hours-of-sleep}{%
\paragraph{Hours of Sleep}\label{hours-of-sleep}}

Participants were asked ``During the past month, on average, how many
hours of \emph{actual sleep} did you get per night''.

\hypertarget{ethnic-group-impermeability}{%
\paragraph{Ethnic group
impermeability}\label{ethnic-group-impermeability}}

We assessed ethnic group impermeability using one item from Mummendey \&
Wenzel (1999). Participants were asked to indicate the extent to which
they agree with the statement (``The current income gap between New
Zealand Europeans and other ethnic groups would be very hard to
change.''; 1 = Strongly Disagree to Strongly Agree).

\hypertarget{income}{%
\paragraph{Income}\label{income}}

Participants were asked ``Please estimate your total household income
(before tax) for the year XXXX''. To stablise this indicator, we first
took the natural log of the response + 1, and then centred and
standardised the log-transformed indicator.

\hypertarget{psychological-distress}{%
\paragraph{Psychological Distress}\label{psychological-distress}}

We measured psychological distress using the Kessler-6 scale (R. ~C.
Kessler et al., 2002), which exhibits strong diagnostic concordance for
moderate and severe psychological distress in large, cross-cultural
samples (R. C. Kessler et al., 2010; Prochaska et al., 2012).
Participants rated during the past 30 days, how often did\ldots{} (1)
``\ldots{} you feel hopeless''; (2) ``\ldots{} you feel so depressed
that nothing could cheer you up''; (3) ``\ldots{} you feel restless or
fidgety''; (4)``\ldots{} you feel that everything was an effort''; (5)
``\ldots{} you feel worthless''; (6) '' you feel nervous?'' Ordinal
response alternatives for the Kessler-6 are: ``None of the time''; ``A
little of the time''; ``Some of the time''; ``Most of the time''; ``All
of the time.''

\hypertarget{life-meaning}{%
\paragraph{Life Meaning}\label{life-meaning}}

We assessed participants' levels of life meaning using two items from
Steger et al. (2006): (1) My life has a clear sense of purpose; (2) I
have a good sense of what makes my life meaningful. Participants
indicated their agreement with these items (1 = Strongly Disagree to 7 =
Strongly Agree).

\hypertarget{life-satisfaction}{%
\paragraph{Life Satisfaction}\label{life-satisfaction}}

We measured life satisfaction with two items adapted from the
Satisfaction with Life Scale (Diener et al., 1985): ``I am satisfied
with my life'' and ``In most ways my life is close to ideal''.
Participants responded on a scale from 1 (Strongly Disagree) to 7
(Strongly Agree).

\hypertarget{lost-job}{%
\paragraph{Lost Job}\label{lost-job}}

Participants were asked to indicate whether they lost their job or had
the principal earner in their household lose job in the last year (1 =
Yes, 0 = No).

\hypertarget{nz-deprivation-index}{%
\paragraph{NZ Deprivation Index}\label{nz-deprivation-index}}

We used the NZ Deprivation Index to assign each participant a score
based on where they live (Atkinson et al., 2019). This score combines
data such as income, home ownership, employment, qualifications, family
structure, housing, and access to transport and communication for an
area into one deprivation score.

\hypertarget{national-well-being-index}{%
\paragraph{National Well-being Index}\label{national-well-being-index}}

We measured participant's sense of national well-being by asking them to
``Please rate your level of satisfaction with the following aspects of
your life and New Zealand: (1) The economic situation in New Zealand,
(2) The social conditions in New Zealand, and (3) Business in New
Zealand'' on a scale from 0 (Completely Dissatisfied) to 10 (Completely
Satisfied) (Tiliouine et al., 2006).

\hypertarget{occupational-prestige-and-status}{%
\paragraph{Occupational Prestige and
Status}\label{occupational-prestige-and-status}}

We assessed occupational prestige and status using the New Zealand
Socio-economic Index 13 (NZSEI-13) (Fahy et al., 2017). This index uses
the income, age, and education of a reference group, in this case the
2013 New Zealand census, to calculate an score for each occupational
group. Scores range from 10 (Lowest) to 90 (Highest). This list of index
scores for occupational groups was used to assign each participant a
NZSEI-13 score based on their occupation.

\hypertarget{parent}{%
\paragraph{Parent}\label{parent}}

Participants were asked ``If you are a parent, what is the birth date of
your eldest child?''.

\hypertarget{living-with-partner}{%
\paragraph{Living with Partner}\label{living-with-partner}}

Participants were asekd ``Do you live with your partner?'' (1 = Yes, 0 =
No).

\hypertarget{perfectionism}{%
\paragraph{Perfectionism}\label{perfectionism}}

We assessed participants' perfectionism using three items from Rice et
al. (2014): (1) Doing my best never seems to be enough; (2) My
performance rarely measures up to my standards; (3) I am hardly ever
satisfied with my performance. Participants indicated the extent to
which they agree with these items (1 = Strongly Disagree to 7 = Strongly
Agree).

\hypertarget{political-orientation}{%
\paragraph{Political Orientation}\label{political-orientation}}

We measured participants' political orientation using a single item
adapted from Jost (2006). Participants were asked to rate how
politically left-wing versus right-wing they see themselves as being (1
= Extremely Left-wing to 7 = Extremely Right-wing)

\hypertarget{power-dependence}{%
\paragraph{Power Dependence}\label{power-dependence}}

Participants' Power dependence was measured using two items: (1) I do
not have enough power or control over important parts of my life; (2)
Other people have too much power or control over important parts of my
life. Participants indicated their agreement with these items (1 =
Strongly Disagree to 7 = Strongly Agree).

\hypertarget{religion-idenfication}{%
\paragraph{Religion Idenfication}\label{religion-idenfication}}

Participants were asked to indicate their religion identification (``Do
you identify with a religion and/or spiritual group?'') on a binary
response (1 = Yes, 0 = No).

\hypertarget{self-respect}{%
\paragraph{Self-Respect}\label{self-respect}}

We assessed participants' levels of self-respect using an item adapted
from Tyler et al. (1996). Participant indicated the extent to which they
agree with the statement (``If they knew me, most NZers would respect
what I have accomplished in life'') on a likert scale (1 = Strongly
Disagree to 7 = Strongly Agree)

\hypertarget{retired}{%
\paragraph{Retired}\label{retired}}

Participants were asked to indicate whether they were retiered or not in
the last year (1 = Yes, 0 = No).

\hypertarget{rumination}{%
\paragraph{Rumination}\label{rumination}}

We used a single item adapted from Nolen-hoeksema \& Morrow (1993) to
measure participants' level of rumination. Participants were asked,
``During the last 30 days, how often did\ldots you have negative
thoughts that repeated over and over?'' and indicated their frequency of
rumination on an ordinal scale (0 = None of The time, 1 = A little of
the time, 2 = some of the time, 3 = most of the time, 4 = all of the
time)

\hypertarget{self-control}{%
\paragraph{Self-Control}\label{self-control}}

Participants were asked to indicate the extent to which they endorse the
two items (``In general, I have a lot of self-control'', ``I wish I had
more self-discipline'') from Tangney et al. (2004). The responses to the
items ranged from 1 (Strongly Disagree) to 7 (Strongly Agree).

\hypertarget{self-esteem}{%
\paragraph{Self-Esteem}\label{self-esteem}}

We measured participants' self-esteem using three items adapted from
Rosenberg (1965). Participants were instructed to circle the number that
best represents how accurately each statement describes them.
Participants responded to the items (``On the whole am satisfied with
myself.'', ``Take a positive attitude toward myself'', ``Am inclined to
feel that I am a failure'') on a likert-type scale (1 = Very inaccurate
to 7 = Very accurate).

\hypertarget{semiretired}{%
\paragraph{Semiretired}\label{semiretired}}

SAME AS RETIRED

\hypertarget{sexual-orientation}{%
\paragraph{Sexual Orientation}\label{sexual-orientation}}

Participants were asked to report their sexual orientation: ``How would
you describe your sexual orientation? (e.g., heterosexual, homosexual,
straight, gay, lesbian, bisexual, etc.)''.

\hypertarget{sexual-satisfaction}{%
\paragraph{Sexual Satisfaction}\label{sexual-satisfaction}}

Participants were asked ``How satisfied are you with your sex life?'' (1
= Not satisfied to 7 = Very satisfied).

\hypertarget{short-form-health}{%
\paragraph{Short-Form Health}\label{short-form-health}}

Participants' subjective health was measured using one item (``Do you
have a health condition or disability that limits you, and that has
lasted for 6+ months?''; 1 = Yes, 0 = No) adapted from Verbrugge (1997).

\hypertarget{smoker}{%
\paragraph{Smoker}\label{smoker}}

We asked participants whether they are currently smoking or not (1 = Yes
or 0 = No), using a single item (``Do you currently smoke?) from Muriwai
et al. (2018).

\hypertarget{spiritual-identification}{%
\paragraph{Spiritual Identification}\label{spiritual-identification}}

Spiritual identification was measured using one item (``I identify as a
spiritual person.'') from Postmes et al. (2013). Participants indicated
their agreement with this item (1 = Strongly Disagree to 7 = Strongly
Agree).

\hypertarget{standard-living}{%
\paragraph{Standard Living}\label{standard-living}}

We measured participants' satisfaction with their standard of living
using an item from the Australian Unity Wellbeing Index (Cummins et al.,
2003). Participants read an instruction (``Please rate your level of
satisfaction with the following aspects of your life and New Zealand.'')
and responded to an item (``Your standard of living'') on a 10-point
scale (0 = completely dissatisfied to 10 = completely satisfied).

\hypertarget{support}{%
\paragraph{Support}\label{support}}

Participants' perceived social support was measured using three items
from Cutrona \& Russell (1987) and Williams et al. (2000): (1) ``There
are people I can depend on to help me if I really need it''; (2) ``There
is no one I can turn to for guidance in times of stress''; (3) ``I know
there are people I can turn to when I need help.'' Participants
indicated the extent to which they agree with those items (1 = Strongly
Disagree to 7 = Strongly Agree). The second item was negatively-worded,
so we reversely recorded the responses to this item.

\hypertarget{living-in-an-urban-area}{%
\paragraph{Living in an Urban Area}\label{living-in-an-urban-area}}

We coded whether they are living in an urban or rural area (1 = Urban, 0
= Rural) based on the addresses provided.

\hypertarget{vengeful-rumination}{%
\paragraph{Vengeful Rumination}\label{vengeful-rumination}}

We assessed participants' vengeful rumination using three items,
respectively adapted from Caprara (1986) and Berry et al. (2005), and
developed for NZAVS: (1) Sometimes I can't sleep because of thinking
about past wrongs I have suffered; (2) I can usually forgive and forget
when someone does me wrong; (3) I find myself regularly thinking about
past times that I have been wronged. Participants indicated their
agreeement with these items (1 = Strongly Disagree to 7 = Strongly
Agree). The values for the sencond item were reversely coded.

\hypertarget{volunteers}{%
\paragraph{Volunteers}\label{volunteers}}

Participants were asked,``Please estimate how many hours you spent doing
each of the following things last week'' and responded to an item
(``voluntary/charitable work''), from (C. G. Sibley et al., 2011).

\hypertarget{subjective-wellbeing}{%
\paragraph{Subjective Wellbeing}\label{subjective-wellbeing}}

We measured participants' subjective wellbeing using three items from
the Australian Unity Wellbeing Index (Cummins et al., 2003): (1) your
health; (2) your future security; (3) your future security. Participants
read an instruction (``The following items assess your current
satisfaction with different aspects of your life and aspects of New
Zealand more generally'') and indicated their satisfaction with those
items (0 = Completely Dissatisfied to 10 = Completely Satisfied).

\pagebreak

\hypertarget{references}{%
\paragraph{References}\label{references}}

\hypertarget{refs}{}
\begin{CSLReferences}{1}{0}
\leavevmode\vadjust pre{\hypertarget{ref-atkinson2019}{}}%
Atkinson, J., Salmond, C., \& Crampton, P. (2019). \emph{NZDep2018 index
of deprivation, user{'}s manual.}

\leavevmode\vadjust pre{\hypertarget{ref-berry_forgivingness_2005}{}}%
Berry, J. W., Worthington Jr., E. L., O'Connor, L. E., Parrott III, L.,
\& Wade, N. G. (2005). Forgivingness, vengeful rumination, and affective
traits. \emph{Journal of Personality}, \emph{73}(1), 183--226.
\url{https://doi.org/10.1111/j.1467-6494.2004.00308.x}

\leavevmode\vadjust pre{\hypertarget{ref-Bulbulia_2015}{}}%
Bulbulia, S., J. A. (2015). Religion and parental cooperation: An
empirical test of slone's sexual signaling model. In \&. V. S. J. Slone
D. (Ed.), \emph{The attraction of religion: A sexual selectionist
account} (pp. 29--62). Bloomsbury Press.

\leavevmode\vadjust pre{\hypertarget{ref-caprara_indicators_1986}{}}%
Caprara, G. V. (1986). Indicators of aggression: The
dissipation-rumination scale. \emph{Personality and Individual
Differences}, \emph{7}(6), 763--769.
\url{https://doi.org/10.1016/0191-8869(86)90074-7}

\leavevmode\vadjust pre{\hypertarget{ref-cummins_developing_2003}{}}%
Cummins, R. A., Eckersley, R., Pallant, J., Vugt, J. van, \& Misajon, R.
(2003). Developing a national index of subjective wellbeing: The
australian unity wellbeing index. \emph{Social Indicators Research},
\emph{64}(2), 159--190. \url{https://doi.org/10.1023/A:1024704320683}

\leavevmode\vadjust pre{\hypertarget{ref-cutrona1987}{}}%
Cutrona, C. E., \& Russell, D. W. (1987). The provisions of social
relationships and adaptation to stress. \emph{Advances in Personal
Relationships}, \emph{1}, 37--67.

\leavevmode\vadjust pre{\hypertarget{ref-diener1985a}{}}%
Diener, E., Emmons, R. A., Larsen, R. J., \& Griffin, S. (1985). The
Satisfaction With Life Scale. \emph{Journal of Personality Assessment},
\emph{49}(1), 71--75.

\leavevmode\vadjust pre{\hypertarget{ref-fahy2017a}{}}%
Fahy, K. M., Lee, A., \& Milne, B. J. (2017). \emph{New Zealand
socio-economic index 2013}. Statistics New Zealand-Tatauranga Aotearoa.

\leavevmode\vadjust pre{\hypertarget{ref-fraser_coding_2020}{}}%
Fraser, G., Bulbulia, J., Greaves, L. M., Wilson, M. S., \& Sibley, C.
G. (2020). Coding responses to an open-ended gender measure in a new
zealand national sample. \emph{The Journal of Sex Research},
\emph{57}(8), 979--986.
\url{https://doi.org/10.1080/00224499.2019.1687640}

\leavevmode\vadjust pre{\hypertarget{ref-gratz_multidimensional_2004}{}}%
Gratz, K. L., \& Roemer, L. (2004). Multidimensional assessment of
emotion regulation and dysregulation: Development, factor structure, and
initial validation of the difficulties in emotion regulation scale.
\emph{Journal of Psychopathology and Behavioral Assessment},
\emph{26}(1), 41--54.
\url{https://doi.org/10.1023/B:JOBA.0000007455.08539.94}

\leavevmode\vadjust pre{\hypertarget{ref-gross_individual_2003}{}}%
Gross, J. J., \& John, O. P. (2003). Individual differences in two
emotion regulation processes: Implications for affect, relationships,
and well-being. \emph{Journal of Personality and Social Psychology},
\emph{85}(2), 348--362. \url{https://doi.org/10.1037/0022-3514.85.2.348}

\leavevmode\vadjust pre{\hypertarget{ref-hagerty1995}{}}%
Hagerty, B. M. K., \& Patusky, K. (1995). Developing a Measure Of Sense
of Belonging: \emph{Nursing Research}, \emph{44}(1), 9--13.
\url{https://doi.org/10.1097/00006199-199501000-00003}

\leavevmode\vadjust pre{\hypertarget{ref-Ministry_of_Health_2013}{}}%
Health, M. of. (2013). \emph{The new zealand health survey: Content
guide 2012-2013}. Princeton University Press.

\leavevmode\vadjust pre{\hypertarget{ref-houkamau_looking_2015}{}}%
Houkamau, C. A., \& Sibley, C. G. (2015). Looking māori predicts
decreased rates of home wwnership: Institutional racism in housing based
on perceived appearance. \emph{{PLoS} {ONE}}, \emph{10}(3), e0118540.
\url{https://doi.org/10.1371/journal.pone.0118540}

\leavevmode\vadjust pre{\hypertarget{ref-hoverd_religious_2010}{}}%
Hoverd, W. J., \& Sibley, C. G. (2010). Religious and denominational
diversity in new zealand 2009. \emph{New Zealand Sociology},
\emph{25}(2), 59--87.

\leavevmode\vadjust pre{\hypertarget{ref-jost_end_2006-1}{}}%
Jost, J. T. (2006). The end of the end of ideology. \emph{American
Psychologist}, \emph{61}(7), 651--670.
\url{https://doi.org/10.1037/0003-066X.61.7.651}

\leavevmode\vadjust pre{\hypertarget{ref-kessler2002a}{}}%
Kessler, R. ~C., Andrews, G., Colpe, L. ~J., Hiripi, E., Mroczek, D.
~K., Normand, S.-L. ~T., Walters, E. ~E., \& Zaslavsky, A. ~M. (2002).
Short screening scales to monitor population prevalences and trends in
non-specific psychological distress. \emph{Psychological Medicine},
\emph{32}(6), 959--976. \url{https://doi.org/10.1017/S0033291702006074}

\leavevmode\vadjust pre{\hypertarget{ref-kessler2010b}{}}%
Kessler, R. C., Green, J. G., Gruber, M. J., Sampson, N. A., Bromet, E.,
Cuitan, M., Furukawa, T. A., Gureje, O., Hinkov, H., Hu, C.-Y., Lara,
C., Lee, S., Mneimneh, Z., Myer, L., Oakley-Browne, M., Posada-Villa,
J., Sagar, R., Viana, M. C., \& Zaslavsky, A. M. (2010). Screening for
serious mental illness in the general population with the K6 screening
scale: results from the WHO World Mental Health (WMH) survey initiative.
\emph{International Journal of Methods in Psychiatric Research},
\emph{19}(S1), 4--22. \url{https://doi.org/10.1002/mpr.310}

\leavevmode\vadjust pre{\hypertarget{ref-mccullough_grateful_2002}{}}%
McCullough, M. E., Emmons, R. A., \& Tsang, J.-A. (2002). The grateful
disposition: A conceptual and empirical topography. \emph{Journal of
Personality and Social Psychology}, \emph{82}(1), 112--127.
\url{https://doi.org/10.1037/0022-3514.82.1.112}

\leavevmode\vadjust pre{\hypertarget{ref-mummendey_social_1999}{}}%
Mummendey, A., \& Wenzel, M. (1999). Social discrimination and tolerance
in intergroup relations: Reactions to intergroup difference.
\emph{Personality and Social Psychology Review: An Official Journal of
the Society for Personality and Social Psychology, Inc}, \emph{3}(2),
158--174. \url{https://doi.org/10.1207/s15327957pspr0302_4}

\leavevmode\vadjust pre{\hypertarget{ref-muriwai_looking_2018}{}}%
Muriwai, E., Houkamau, C. A., \& Sibley, C. G. (2018). Looking like a
smoker, a smokescreen to racism? Māori perceived appearance linked to
smoking status. \emph{Ethnicity \& Health}, \emph{23}(4), 353--366.
\url{https://doi.org/10.1080/13557858.2016.1263288}

\leavevmode\vadjust pre{\hypertarget{ref-nolen-hoeksema_effects_1993}{}}%
Nolen-hoeksema, S., \& Morrow, J. (1993). Effects of rumination and
distraction on naturally occurring depressed mood. \emph{Cognition and
Emotion}, \emph{7}(6), 561--570.
\url{https://doi.org/10.1080/02699939308409206}

\leavevmode\vadjust pre{\hypertarget{ref-postmes_single-item_2013}{}}%
Postmes, T., Haslam, S. A., \& Jans, L. (2013). A single-item measure of
social identification: Reliability, validity, and utility. \emph{The
British Journal of Social Psychology}, \emph{52}(4), 597--617.
\url{https://doi.org/10.1111/bjso.12006}

\leavevmode\vadjust pre{\hypertarget{ref-prochaska2012a}{}}%
Prochaska, J. J., Sung, H.-Y., Max, W., Shi, Y., \& Ong, M. (2012).
Validity study of the K6 scale as a measure of moderate mental distress
based on mental health treatment need and utilization: The K6 as a
measure of moderate mental distress. \emph{International Journal of
Methods in Psychiatric Research}, \emph{21}(2), 88--97.
\url{https://doi.org/10.1002/mpr.1349}

\leavevmode\vadjust pre{\hypertarget{ref-rice_short_2014}{}}%
Rice, K. G., Richardson, C. M. E., \& Tueller, S. (2014). The short form
of the revised almost perfect scale. \emph{Journal of Personality
Assessment}, \emph{96}(3), 368--379.
\url{https://doi.org/10.1080/00223891.2013.838172}

\leavevmode\vadjust pre{\hypertarget{ref-Rosenberg1965}{}}%
Rosenberg, M. (1965). \emph{Society and the adolescent self-image}.
Princeton University Press.

\leavevmode\vadjust pre{\hypertarget{ref-sengupta2013}{}}%
Sengupta, N. K., Luyten, N., Greaves, L. M., Osborne, D., Robertson, A.,
Brunton, C., Armstrong, G., \& Sibley, C. G. (2013). Sense of Community
in New Zealand Neighbourhoods: A Multi-Level Model Predicting Social
Capital. \emph{New Zealand Journal of Psychology}, \emph{42}(1), 36--45.

\leavevmode\vadjust pre{\hypertarget{ref-sibley2012}{}}%
Sibley, \&. B., C. G. (2012). Healing those who need healing: How
religious practice affects social belonging. \emph{Journal for the
Cognitive Science of Religion}, \emph{1}, 29--45.

\leavevmode\vadjust pre{\hypertarget{ref-sibley2011}{}}%
Sibley, C. G., Luyten, N., Purnomo, M., Mobberley, A., Wootton, L. W.,
Hammond, M. D., Sengupta, N., Perry, R., West-Newman, T., Wilson, M. S.,
McLellan, L., Hoverd, W. J., \& Robertson, A. (2011). The Mini-IPIP6:
Validation and extension of a short measure of the Big-Six factors of
personality in New Zealand. \emph{New Zealand Journal of Psychology},
\emph{40}(3), 142--159.

\leavevmode\vadjust pre{\hypertarget{ref-steger_meaning_2006}{}}%
Steger, M. F., Frazier, P., Oishi, S., \& Kaler, M. (2006). The meaning
in life questionnaire: Assessing the presence of and search for meaning
in life. \emph{Journal of Counseling Psychology}, \emph{53}(1), 80--93.
\url{https://doi.org/10.1037/0022-0167.53.1.80}

\leavevmode\vadjust pre{\hypertarget{ref-stronge_facebook_2015}{}}%
Stronge, S., Greaves, L. M., Milojev, P., West-Newman, T., Barlow, F.
K., \& Sibley, C. G. (2015). Facebook is linked to body dissatisfaction:
Comparing users and non-users. \emph{Sex Roles: A Journal of Research},
\emph{73}(5), 200--213. \url{https://doi.org/10.1007/s11199-015-0517-6}

\leavevmode\vadjust pre{\hypertarget{ref-tangney_high_2004}{}}%
Tangney, J. P., Baumeister, R. F., \& Boone, A. L. (2004). High
self-control predicts good adjustment, less pathology, better grades,
and interpersonal success. \emph{Journal of Personality}, \emph{72}(2),
271--324. \url{https://doi.org/10.1111/j.0022-3506.2004.00263.x}

\leavevmode\vadjust pre{\hypertarget{ref-tiliouine2006}{}}%
Tiliouine, H., Cummins, R. A., \& Davern, M. (2006). Measuring Wellbeing
in Developing Countries: The Case of Algeria. \emph{Social Indicators
Research}, \emph{75}(1), 1--30.
\url{https://doi.org/10.1007/s11205-004-2012-2}

\leavevmode\vadjust pre{\hypertarget{ref-tyler_understanding_1996}{}}%
Tyler, T., Degoey, P., \& Smith, H. (1996). Understanding why the
justice of group procedures matters: A test of the psychological
dynamics of the group-value model. \emph{Journal of Personality and
Social Psychology}, \emph{70}(5), 913--930.
\url{https://doi.org/10.1037/0022-3514.70.5.913}

\leavevmode\vadjust pre{\hypertarget{ref-verbrugge1997}{}}%
Verbrugge, L. M. (1997). A global disability indicator. \emph{Journal of
Aging Studies}, \emph{11}(4), 337--362.
\url{https://doi.org/10.1016/S0890-4065(97)90026-8}

\leavevmode\vadjust pre{\hypertarget{ref-williams_cyberostracism_2000}{}}%
Williams, K. D., Cheung, C. K. T., \& Choi, W. (2000). Cyberostracism:
Effects of being ignored over the internet. \emph{Journal of Personality
and Social Psychology}, \emph{79}(5), 748--762.
\url{https://doi.org/10.1037/0022-3514.79.5.748}

\end{CSLReferences}



\end{document}
